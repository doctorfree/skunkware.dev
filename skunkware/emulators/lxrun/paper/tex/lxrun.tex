\documentclass[letterpaper]{article}
\usepackage{linuxdoc-sgml}
\usepackage{qwertz}
\usepackage{url}
\usepackage[latin1]{inputenc}
\usepackage{epsf}
\usepackage{null}
\def\addbibtoc{
\addcontentsline{toc}{section}{\numberline{\mbox{}}\relax\bibname}
}%end-preamble
\setcounter{page}{1}
\title{Linux Emulation for SCO}
\author{Ronald Joe Record
{\ttfamily \url{(rr@sco.com)}}, {\bfseries SCO}\\ 
Michael Hopkirk
{\ttfamily \url{(hops@sco.com)}}, {\bfseries SCO}\\ 
Steve Ginzburg
{\ttfamily \url{(steven@ugcs.caltech.edu)}}, {\bfseries California Institute of Technology}}
\date{May 8, 1998}
\abstract{This paper describes some of the rationale and implementation
decisions for lxrun, an Intel Linux emulator, plus how to get, build, configure and run it; what some of the current and future development
issues and enhancements are and the current status of the project.
There is not much difference between the execution environment required by Linux binaries and binaries for other Intel UNIX platforms, the main one being the way in which system calls are handled. For example, in Linux an "int \$0x80" instruction is used, which jumps to the system-call-handling portion of the Linux kernel. On SCO systems, "int \$0x80" is an unused vector and therefore causes a general protection trap resulting in a SIGSEGV signal. Lxrun intercepts these signals and calls the SCO equivalent of the system call that the Linux program attempted. It also remaps some ioctls, flags, return values and error codes. Using lxrun, a Linux binary can be run on a non-Linux platform with little performance penalty. Lxrun can also take advantage of the lower overhead in some Linux libraries, occasionally resulting in improved performance over native binaries. No kernel modifications are necessary.}


\begin{document}
\maketitle
\tableofcontents

\section{Introduction}




\subsection{What is lxrun ?}

Lxrun is an emulator that allows the execution of Intel Linux binaries on
Intel UNIX\rcircle{} platforms. The currently supported platforms are
SCO OpenServer 5 (UNIX SVR3), SCO UnixWare 2.x (UNIX SVR4) and 
SCO UnixWare 7 (UNIX SVR5).

Lxrun works by remapping system calls on the fly. Since lxrun does its work 
at the system call level, it requires copies of the Linux dynamic loader 
(ld-linux.so.*) and whatever Linux shared libraries are required by the 
program being run. The current development release of lxrun consists of
approximately 6000 lines of code (146 Kilobytes).

Most programs that do not rely on Linux-specific quirks
or deal directly with hardware should work under lxrun.
Users of lxrun have reported success with raplayer (RealAudio client),
xquake, the StarOffice suite, gcc (the GNU C compiler), smbclient,
the AC3D modeller and a myriad of smaller applications and utilities.




\subsection{Why should I use lxrun ?}

The original impetus for writing lxrun was to be able to run Netscape
Navigator 1.x on SCO OpenServer. When Netscape ported Navigator to SCO 
platforms, the need was obviated. Later, work was resumed in an attempt to 
run Adobe Acrobat Reader on OpenServer. The body of system call mappings in 
lxrun grew gradually as users modified it to work with more and more 
applications. By this process, lxrun became quite robust, able to handle 
sophisticated X11, audio, and networking applications.

Lxrun promotes the interoperability of Linux with other UNIX and
UNIX-like platforms. This improves the user's ability to
combine the best aspects of available operating systems. For
example, it allows a user to combine Linux's large freeware
application base with SCO's:

\begin{itemize}
\item Large robust filesystem support
\item Large installed base
\item Renowned and Industry-proven scalability
\item Enterprise applications base
\item Installed Enterprise server deployment
\item Network Computer / "Any Client" deployment of applications
\item Ability to distribute and execute applications via 
\nameurl{http://tarantella.sco.com/}{Tarantella}
\item Robust DOS/Windows '95 emulation capabilities
\item \nameurl{http://www.sco.com/offers/}{Free UNIX}
offer makes for extremely inexpensive robust UNIX workstation
\item SVR5 enhancements
\begin{itemize}
\item Up to 64GB of main memory 
\item 1 TB File Systems and disk partitions 
\item 1 TB file sizes 
\item Up to 76,800 TB of total storage supported 
\item Up to 32 CPUs supported 
\item Support for I2O Peripherals 
\item 64-bit technology
\item Advanced Clustering Support for up to 4 nodes 
\item Hot Pluggable Disk and Tape Drives 
\item Disk Mirroring and Spanning 
\item Hot Plug PCI 
\item Multi-path I/O 
\item Real-time Data Management 
\item UPS Monitoring
\item Centralized, on-line backup
\item Journaling File Systems for improved data recovery
\end{itemize}
\end{itemize}


Every UNIX vendor could construct such a "Why you should use my
platform" list.  Lxrun allows the user to leverage the large base
of precompiled Linux software without restricting the choice of platform.




\section{Implementation}

The emulator needed to handle the following major issues (detailed below):
system call handling along with remapping of arguments, flags, return values
and error codes; library pathname lookup and remapping; ioctl mappings for
the various major devices and differences in the tty subsystems.




\subsection{Design goals}

\begin{itemize}
\item Allow execution of Intel Linux binaries on other Intel UNIX platforms
\item Exist entirely in user-space (no kernel modifications)
\item No modification of the Linux binaries or libraries required
\end{itemize}



\subsection{Difference in syscall handling}

System call handling is implemented differently between Linux and Intel
UNIX System V platforms; that is each uses a different instruction
to implement the switch to kernel mode.

Intel Unix System V uses a "lcall \$0x07".  In Linux an "int \$0x80" software
interrupt instruction is used, which jumps to the  system-call-handling 
portion of the Linux kernel. On SCO systems, "int \$0x80" is an unused vector 
and therefore causes a general protection trap resulting in a SIGSEGV
signal. Lxrun intercepts these signals and calls the SCO equivalent of
the system call that the Linux program attempted. In many cases this is
a direct map or call of the equivalent native system call, in other cases
some mapping or translation of arguments passed in and flags and error codes
passed out is required. Where there is no equivalent system call available
on the emulated system or the equivalent syscall mapping has not been
implemented the system call fails and returns an errorcode of ENOSYS.




\subsection{Pathname for loading Linux libraries }

Because lxrun works at the system call level, any Linux shared
libraries required by the application must be present in the
emulation environment.  This leads to a possible filename-space
conflict between native and Linux binaries.  To resolve this
problem, lxrun remaps any pathnames beginning with /lib,
/usr/lib, /usr/local/lib, or /usr/X11R6/lib by prepending them
with a "Linux root" path.  This path is specified at compile
time and can be overridden by setting the LINUX\_ROOT environment
variable. This remapping allows the user to install a set of
Linux libraries in a separate directory hierarchy from the
native system libraries, thus avoiding conflicts.




\subsection{Device major number mappings}

The arguments to an ioctl call alone do not provide enough
information to remap the command number correctly.  This is
because the same command number can have different meanings to
different drivers.

Lxrun works around this problem by maintaining a mapping from
open file descriptors to drivers.  On the first ioctl call to a
new file descriptor, lxrun determines the associated major
device number.  It compares this with a table of drivers and
major device numbers set up at run-time.  This mapping is then
cached to improve performance on future ioctl calls to that file
descriptor.  Lxrun can then take the driver into consideration
when remapping ioctl command numbers.




\subsection{Differences in kernel tty systems}

The following is a discourse on lxrun tty handling from Robert Lipe
\url{(robertl@dgii.com)}, the
principal author of lxrun's tty handling code:

"The kernel tty systems are very different. The user level tty systems
are actually quite similar, fortunately for lxrun. Most of the members
of things like termio and struct termios are the same size, alignment, 
and offset. Even most of the bitfields fall into place. While
it would have been much safer to do
\par
\addvspace{\medskipamount}
\nopagebreak\hrule
\begin{verbatim}
        if ( (lx_tio->c_cflag & LX_CBAUD) == LX_B50)
                tio->c_cflag |= B50;
        if (lx_tio->c_iflag & LX_ICANON) 
                tio->c_iflag |= ICANON;
\end{verbatim} 
\nopagebreak\hrule 
\addvspace{\medskipamount}

and repeat this for each of a couple hundred flags, the reality was that
it generated horrible code that would have performed poorly and been
a nightmare to maintain. 

The only sticky spot was that Linux has four distinctly separate
members in c\_cc (the control character array) for VMIN, VTIME, EOF, 
and EOL. Most System V's including OpenServer (I can't recall what SVR[45] 
does here) use the same offset and multiplex this into two bytes. In reality, 
since they can never both be active at the same time (if ICANON is set, EOF 
and EOL are used. If ICANON is clear, VMIN, VTIME are used) this doesn't turn
out to be much of a issue.

[Since] TCGETA and TCSETA are by far the most frequently used, I implemented 
them first (and their derivatives that wait and flush and the derivatives of 
each of those for POSIX struct termios). With those 8 in hand, I started firing 
up applications that were known to do wierd things to the tty. When elvis 
(the Linux vi binary I had at the time) worked well enough, I submitted the 
changes. Then, I just used more and more applications and picked up a few 
stragglers like FIONREAD that mapped very simply. 

Are there hazards in any of this? Certainly. Each system has a few
bits in each of the available ioctls that don't exist in the other
system. There is some overlap. We haven't seen any real world
failures becuase of this. The reality seems to be that the programs
that drive serial ports to the crazy edge don't make sense to emulate
anyway. For example, someone once asked me why Linux ecu (a kermit-like
program) didn't work well. Since source for it is available and it
supports OpenServer just fine, use the native binary instead. Someone
once asked me about running Linux ppp, but that's similarly nonsensical,
though doomed to failure for different reasons."




\subsection{Miscellaneous issues}

\begin{itemize}
\item  For local displays, SCO OpenServer uses ":0.0" and UnixWare uses
"unix:0.0".  The former causes Linux X11 binaries to try to
use shared memory, which won't work with the native X server.
The latter causes a name lookup for the machine "unix" which
probably fails. These two cases are detected and the DISPLAY
is set to contain a valid machine name instead.
\item  Documentation is supplied in the {\ttfamily doc} subdirectory of the lxrun
source. This currently consists of this document, a FAQ, a document
describing how to get the Linux StarOffice suite to run, and the system
call mapping table.
\item  Various packaging conveniences have been supplied. For instance, the
necessary Linux shared libraries have been re-packaged in the native
installation format. A shell script, lxfront, useful in the execution
of Linux a.out and statically-linked ELF binaries, has been supplied.
To use lxfront:
\begin{itemize}
\item  Put lxfront and any Linux a.out or statically-linked ELF 
binaries in /usr/local/linux/bin
\item  Put lxrun in /usr/local/bin
\item  Create links in /usr/local/bin to lxfront with the same 
names as your Linux binaries.
\end{itemize}

Executing these links should invoke lxrun on your Linux program.
\end{itemize}





\section{Development Issues}






\subsection{Direct execution of Linux ELF binaries\label{sec-direct}}

Rather than running Linux binaries with the lxrun front-end, it is possible
to turn lxrun into a program interpreter, ld-linux.so.* - the Linux runtime 
linker. The path of the interpreter is actually embedded in every dynamic 
linked ELF executable - for standard UnixWare / OpenServer ELF it is
/usr/lib/libc.so.1, and for Linux it is /lib/ld-linux.so.1 (or, ld-linux.so.2
for executables linked with GNU libc v2). 

The direct execution of Linux binaries in this manner is in the latest 
development releases of lxrun. At the time of this writing, support for
this model of execution is available on UnixWare platforms only. SCO OpenServer
places some rather strict limitations on what a program interpreter can be.
Overcoming these restrictions is a current development topic.

The following discourse, from Mike Davidson
\url{(md@sco.com)} the author of lxrun, 
details some of the issues involved in the direct execution model:

"It's actually quite simple to build a version of lxrun which
can be used as the initial program interpreter for Linux binaries.
There are, however, a couple of problems:
\begin{itemize}
\item  runtime linkers are usually ELF shared objects
which don't have a fixed load address, and which have
to do some rather delicate data relocations when they
first start up - I avoided this issue by making the
ld-linux.so.* version of lxrun be an actual ELF executable
(rather than a shared library) bound to a fixed address
which does not conflict with the addresses used by normal
Linux binaries - while this is a bit of a kludge it works
perfectly well
\item  the interpreter program loaded by the kernel cannot itself
require an interpreter - what this means is that if the
program interpreter wants to do any dynamic loading of
shared libraries it has to do it for itself - this isn't
too much of a problem on OpenServer since you can just
do a static link of everything that lxrun needs into a
single binary - unfortunately this doesn't work on UnixWare 7
since libsocket is {\itshape only\/} available as a shared library.
\end{itemize}

I think that there is a way round this, but I haven't had time to
try it out yet. Essentially it looks something like this:
\begin{itemize}
\item   ld-linux.so.* is created as a dynamic linked shared object
with appropriate dependencies on other shared libraries
(ie at least /usr/lib/libc.so.1 and /usr/lib/libsocket.so.1)
\item   when ld-linux.so.* is loaded by the kernel and gets control,
the {\itshape first\/} thing that it does is to fix up enough of it's
own data relocations in order to be able to run
\item   once that is done, it maps /usr/lib/libc.so.1 (ie the normal
system runtime linker and standard library) into memory,
fakes up a suitable aux vector and invokes the normal
runtime linker, while pretending to be a normal executable
program (this involves a lot of trickery - you have to fake up
an appropriate set of program headers to give to the runtime
linker, you have to fix up entries in .dynamic, .dynsym and
.rel.* to reflect the actual address that ld-linux.so.* is
actually loaded at, and you have to provide a fake entry
point address so that the system runtime linker will give
control back to ld-linux.so.* in the right place ....)
\item   if all of that works, then ld-linux.so.* proceeds as normal,
and now maps in the Linux runtime linker and passes control
to it
\end{itemize}

I realise that this all sounds hideously complicated, but all of the
alternatives that I can think of are worse in some way.

At first I thought that we could avoid this by just porting a
version of the Linux runtime linker to run native on UNIX, but
this looks like it may be more trouble than it is worth - while
the Linux runtime linker is quite well written it has to deal
with some rather unpleasant limitations in the GNU tools - in
particular it looks as if GNU ld doesn't support the equivalent
of our -Bsymbolic option which makes writing the startup code
for a runtime linker almost impossible (in fact the Linux runtime
linker startup routine is just one massive function that uses
nothing but local variables and which does all of it's system
calls with chunks of inline assembler)."




\subsection{SIGSEGV vs. software interrupt kernel module}

Consideration was given to implementing a software interrupt kernel module
rather than relying on the "int \$0x80" segmentation violation. Rather than
sacrifice the elegance and portability of a "non-kernel" Linux emulation
strategy, Mike Davidson has suggested that:
\begin{quotation}
This probably isn't really necessary. Assuming that we are really
only interested in Linux ELF (and {\itshape not\/} Linux a.out) we can use
the dynamic linker to "preload" the Linux runtime compatibility
library in such a way that almost all of the system calls will be
intercepted and handled directly by the compatibility library 
without ever going down to an actual "int \$0x80" instruction.
\end{quotation}





\subsection{Performance enhancements}

Thus far, performance has not been an issue as very little negative
impact has been detected.  The main cost of running a Linux application under
lxrun is the overhead of catching the segmentation violation (int \$0x80),
fixing up the structures/errors/returns, and mapping the system call.
Since much of what lxrun does is done in the normal course of executing a 
system call natively, the SIGSEGV intercept is the main overhead.

In order to avoid catching SIGSEGV for every system call, current lxrun
development plans on implementing a "pre-load" of the "Linux runtime 
compatibility library" (see the previous subsection). With the direct 
execution of Linux ELF binaries described above, it is possible to pre-load 
this library on startup. That is, the binary has already
been identified as a Linux ELF binary since it is attempting to load
/lib/ld-linux.so.1. This "fake" program interpreter knows it will have to
map system calls so, rather than waiting for the SIGSEGV to trigger the 
mapping, the program interpreter can "pre-map". Thus, system calls made by
the Linux ELF binary under the control of such a program interpreter would
not cause general protection traps. In this scenario, nearly all of the 
performance overhead of running Linux binaries with lxrun is eliminated.




\section{System Call Mapping}

Lxrun emulates or maps most commonly-used system calls for which native 
equivalents exist. Unimplemented calls return an error indication and set 
ENOSYS.

The files doc/SysCallTable* list tables of supported, partially supported, 
and unsupported system calls. The SyscallScript utility run (against the 
source) will regenerate the text and HTML versions of this file. An on-line 
copy of the currently supported system calls is available at 
\nameurl{http://www.sco.com/skunkware/emulators/SyscallTable.html}{}.




\subsection{Directly mapped system calls}

The following system calls are mapped directly:

{\itshape exit() fork() creat() link() unlink() chdir() time() mknod() chmod() chown() 
lseek() getpid() setuid()
getuid() stime() alarm() pause() utime() access() nice() sync() rename() 
mkdir() rmdir() dup() times()
setgid() getgid() geteuid() getegid() setpgid() umask() chroot() dup2() 
getppid() getpgrp() setsid()
setreuid() setregid() sethostname() gettimeofday() settimeofday() symlink() 
readlink()
truncate() ftruncate() fchmod() fchown() setitimer() getitimer() fsync() 
setdomainname()
getpgid() fchdir() sysfs() getdents() readv() writev() getsid()\/}


\subsection{System calls with a non-stub emulation function}

For the following system calls, lxrun either provides some
remapping of arguments, return values, and error codes, or in
cases where an analogous native call does not exist, emulates
the call using native library functions.

{\itshape nosys() read() write() open() close() waitpid() execve() oldstat() ptrace() 
oldfstat() kill() pipe()
brk() signal() fcntl() olduname() sigaction() sgetmask() ssetmask() 
sigsuspend() sigpending()
setrlimit() getrlimit() getgroups() setgroups() old\_select() oldlstat() 
uselib()
readdir() mmap() munmap() getpriority() setpriority() socketcall() syslog() 
stat() lstat()
fstat() uname() iopl() wait4() sysinfo() ipc() sigreturn() newuname() 
mprotect() sigprocmask()
personality() \_llseek() select()\/}




\subsection{Partial implementations}

The following system calls are partially emulated:

{\itshape ioctl() ioperm() fdatasync()\/}




\subsection{Unimplemented system calls}

The list of unimplemented system calls is as follows:

{\itshape mount() umount() getrusage() swapon() reboot() statfs() fstatfs() vhangup() 
idle() vm86() swapoff()
clone() modify\_ldt() adjtimex() create\_module() 
init\_module()
delete\_module() get\_kernel\_syms() quotactl() bdflush() 
setfsuid()
setfsgid() flock() msync() sysctl() mlock() munlock() mlockall() munlockall()
sched\_setparam() sched\_getparam() sched\_setscheduler()
sched\_getscheduler() sched\_yield()
sched\_get\_priority\_max()
sched\_get\_priority\_min()
sched\_rr\_get\_interval()\/}




\section{Getting started}




\subsection{How do I get lxrun ?}

Lxrun is currently distributed as a component of {\bfseries SCO Skunkware}
\nameurl{http://www.sco.com/skunkware/}{},
a free CD-ROM containing hundreds of megabytes of pre-compiled and pre-packaged 
software for SCO platforms. A Skunkware CD-ROM can be obtained via
\nameurl{http://www.sco.com/offers/}{} and, beginning in 1998, all operating
systems released by SCO will contain a Skunkware CD-ROM in the shrink-wrapped
product. Lxrun may also be obtained via the SCO Skunkware web site at 
\nameurl{http://www.sco.com/skunkware/emulators/}{}.
All SCO Skunkware software is freely redistributable.




\subsection{Building lxrun from source}

Detailed instructions on building lxrun from source, configuring the build, 
installing the emulation system and additional run-time components, installing 
a Linux binary, runtime environment variables, and error messages are contained
in the file {\ttfamily INSTALL} in the lxrun source distribution and on-line in
\nameurl{http://www.sco.com/skunkware/emulators/lxrun/FAQ.html}{the lxrun FAQ}.

The lxrun source distribution contains a Makefile with support for compilation
on SCO OpenServer 5, UnixWare 2.x, and UnixWare 7. The Makefile uses the
output of "uname -r" to determine the platform. As additional
platform support is added this will need to be augmented or the configuration
modified to use autoconfig.

To build lxrun on one of the supported platforms, it is only necessary to
issue the command "make". The command "make install" will both build lxrun
and copy the resulting binaries and documentation into \$(DESTDIR) which
is set by default to the ./dist directory.

As a convenience this distribution includes a script called 
lxfront which can be used with a symbolic link (see below) to provide 
a wrapper around the invocation of lxrun the Linux binary name
allowing them to be run directly.

Starting with lxrun 0.9.0 the build of lxrun will produce an ld-linux.so.1
as well as the lxrun binary. The ld-linux.so.1 is installed in /lib on the
target system and provides support for direct execution of Linux binaries,
thus deprecating the need for the lxrun binary front-end except for the
execution of Linux a.out binaries.

The lxrun source has the following capability ifdefs:
\begin{itemize}
\item {\bfseries DEF\_LINUX\_ROOT} -
Default root directory name for lxrun to searching under for any
native Linux files. 
Specifically when a Linux library tries to load a dynamic library, lxrun
remaps the pathname to somewhere below this directory.  
The value specified here becomes the internal default which can be 
overridden by the LINUX\_ROOT environment variable.
Set in Makefile with make macro LXROOT.
Default value is "/usr/local/linux"
\item {\bfseries TRACE} -
Flag for making a version of lxrun that emits system call traces 
(to file /tmp/lxrun.nnn (where "nnn" is the process pid.))
This can be used for tracing a binaries use of an unimplemented 
system call or other runtime problems. Its not enabled bu default 
since it slows down the operation of lxrun and produces large log files.
(build should be augmented to make a variantly named lxrun binary with this 
on regardless).
Set in Makefile with make macro TRACE.
Default is disabled (off)
\item {\bfseries ELF\_DEBUG} -
Flag to enable output to stderr of debug traces for the 
ELF loader capability of lxrun. Outputs ELF Header information,
interpreter remap values and open, load, mmap, mprotect values/status
(should be modified to be integrated with TRACE logging)
Set in Makefile via make macro DBG.
Default is disabled (off)
\item {\bfseries UNIDIRECTIONAL\_PIPES} -
Flag to enable use of unidirectional pipes (pipe() ) 
instead of bidirectional pipes (socketpair()) for the pipe()
syscall emulation on platforms where pipes are not bidirectional.
Not referenced in Makefile.
Default is disabled (emulation will use bidirectional pipes)
\item {\bfseries NO\_DISPLAY\_HACK} -
Flag to disable remapping of DISPLAY variable from a local server 
specification to a full hostname specification.
This remapping is done (by default) to address some 
problems with local connections on OSr5 (at least) to some X servers.
Not referenced in Makefile.
Default is disabled (remapping will be done)
\item {\bfseries LXRUN\_AUTO\_PATH\_BEHAVIOR} -
if enabled makes lxrun search for the Linux binary to be run in the 
normal PATH rather than in the (expected) absolute pathname given.
Not referenced in Makefile.
Default is disabled (off)
\item {\bfseries ELFMARK\_HACK} -
enables detection of binaries marked (with elfmark) as Linux binaries
(mark value "LXRN" as an unsigned long).
Not referenced in Makefile.
Default is disabled (off) - status is experimental.
\end{itemize}


Platform defines for OpenServer 5 (OSR5), UnixWare 2.x (UNIXWARE) 
and UnixWare 7 (GEMINI) are automatically setup in the Makefile.




\subsection{Installing the emulation system}

Lxrun expects to find all its (Linux) library files in a normal root hierarchy
rooted under a single place called the LINUX\_ROOT.
Unless respecified in the build this defaults internally in lxrun to
/usr/local/linux.

You can respecify or change it at runtime with the environment variable
LINUX\_ROOT (wherever it ends up this must be the place the Linux 
libraries are placed under).

"make install" will install the built binary (lxrun) into /usr/local/bin,
lxfront into \$LINUX\_ROOT/bin, and the lxrun program interpreter
ld-linux.so.1 into /lib. The HTML documents describing lxrun are placed
in /usr/local/man/html/lxrun.




\subsection{Installing Linux applications}



With lxrun 0.9.0 and later, Linux ELF binaries can be installed anywhere in
the standard execution path (e.g. /usr/local/bin). See section
\ref{sec-direct} {(Direct execution of Linux ELF binaries)}
for details on how this is done. Further, Linux applications distributed in 
RPM format can be installed using either a native RPM port 
or the Linux RPM run under the control of lxrun. Some additional arguments
to RPM may be necessary. For instance, a native port of RPM for SCO OpenServer 
is available at \nameurl{http://www.sco.com/skunkware/osr5/sysadmin/rpm/}{}).
Using the SCO OpenServer RPM it is necessary to invoke RPM as follows:
\begin{verbatim}
    rpm --nodeps --ignorearch --ignoreos --prefix /usr/local ...
\end{verbatim}

The Skunkware distribution of RPM for OpenServer includes a shell script
front-end {\ttfamily rpm4sco} which inserts these arguments for you.

Linux a.out and statically linked ELF binaries should be installed in 
/usr/local/linux/bin and symbolic links by the name of the binary created 
from /usr/local/bin to the lxfront shell script in /usr/local/linux/bin.




\subsection{Error messages}

{\ttfamily linuxemul: fatal error: program load failed: No such file or directory}

Indicates that the Linux binary couldn't run 
(either lxrun could't find the Linux binary or the Linux binary 
couldn't find the dynamic linker)
It probably means your LINUX\_ROOT environment variable isn't set up 
correctly or you don't have the required minimum Linux libraries.

{\ttfamily progname: can't load library 'some\_library\_name.so'}

Indicates you're missing a shared library that is needed to
run a particular binary.  You can either try to find a
compiled version of the library from a Linux ftp site
(such as \nameurl{ftp://sunsite.unc.edu/pub/Linux/libs/}{}) or if you
have access to a running Linux system, you can copy the
library directly.  You should put the library in
\$LINUX\_ROOT/lib on your host system 
(/usr/localLinuxlib by default on a SCO system ).

{\ttfamily myprog: can't resolve symbol '\_\_iob'\\ 
myprog: can't resolve symbol '\_\_iob'\\ 
myprog: can't resolve symbol '\_\_ctype'}

Indicates that the Linux dynamic loader found a native SCO library
and is using it instead of the corresponding Linux binary.  
(You can find out exactly which library is causing the problem by
examining the lxrun.log file produced by a debugging version of
lxrun.)

This will only occur if you have native libraries
installed that have the same names as a dependant Linux binary.
If you have XFree86 installed, the /usr/X11R6/lib libraries are
common culprits.

The best solution is to make sure no native libraries are
available anywhere under the directory pointed to by
\$LINUX\_ROOT.




\subsection{If you come across an unsupported binary}

\begin{enumerate}
\item  Go to \nameurl{http://www.sco.com/skunkware}{} and make
sure you have the most recent version of lxrun.  If not,
download the latest one and try it.  We are updating
lxrun with new system calls all the time.
\item  Recompile lxrun with the TRACE option enabled.  (This
requires modifying one line in the Makefile.)  This will
cause lxrun to produce a history of all system calls
used by the binary as it was run (similar to the
"truss" and "trace" commands).  The trace dump will be
created in a file called "/tmp/lxrun.nnn" where "nnn" is
the process id.
\item  Try to narrow down exactly which system call failed.
Most likely, the failure will be due to a system call
that has not yet been implemented in lxrun.
\item  Implement the system call mapping.  This is usually
pretty easy to do.  The vast majority of lxrun's code
does mappings of this sort, so you can pick out almost
any source file to see how it is done.  Chances are, the
system call you need to remap is already in one of the
lxrun source files, but its code looks something like
this:

int lx\_flock()          $\{$ errno=ENOSYS; return -1; $\}$    

This means that you're the first person who has gotten
around to mapping that particular system call.
\item  After making your modification, recompile lxrun and
see if it works.  You may have to remap more than one
system call to get your binary working!
\item  E-mail your changes to 
\url{skunkware@sco.com}.
This way, we can put your changes into the next release of lxrun.
\item  If steps 1-5 seem beyond your programming
ability, contact 
\url{skunkware@sco.com}.
and maybe one of the
Skunkware team will have time to give you a hand with it.
Make sure to tell us exactly what program you're having
trouble with, and if possible, tell us where you got it.
\end{enumerate}





\section{Web Presence}

The lxrun web site is at 
\nameurl{http://www.sco.com/skunkware/emulators/lxrun/}{}.
Lxrun source is available at
\nameurl{ftp://ftp.sco.com/skunkware/src/emulators}{}.

Any source changes made (augmentation or bug fixes) doc changes
feature requests, questions or problem reports should be mailed to 
\url{skunkware@sco.com}




\section{Authors and Contributors}

The original author of lxrun was Michael Davidson, an engineer at SCO.
Major initial followup work was done by Robert Lipe and Steve Ginzburg.
Andrew Gallatin ported it to Solaris/x86 and the rest of the cast includes
Bela Lubkin, John W. Temples, Mike Hopkirk, Ralf Gelfand, 
Ronald Joe Record and Udo Monk.

Contributors to this document included Michael Davidson, Michael Hopkirk,
Robert Lipe, Steve Ginzburg and the principal author - Ronald Record.




\section{About This Document}

This document was created using SGML-Tools 1.0.6 in conjunction with TeX, 
Version 3.14159 (Web2C 7.2) running on an SCO UnixWare 7 platform. 

The source to this document is maintained at
\nameurl{http://www.sco.com/skunkware/emulators/lxrun/lxrun.sgml}{}.
A Makefile and formatted varieties of this document are also available at
\nameurl{http://www.sco.com/skunkware/emulators/lxrun/}{}. For instance,
you will find a postscript version at
\nameurl{http://www.sco.com/skunkware/emulators/lxrun/ps/lxrun.ps}{}.

This document is Copyright (C) 1998 by Ronald Joe Record.
All rights reserved. Permission to use, copy, and distribute this 
document for any purpose and without fee is hereby granted, provided 
that the above copyright notice appear in all copies and that both that 
copyright notice and this permission notice appear in supporting documentation.



\end{document}
