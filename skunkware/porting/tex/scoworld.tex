\documentclass[letterpaper]{article}
\usepackage{linuxdoc-sgml}
\usepackage{qwertz}
\usepackage{url}
\usepackage[latin1]{inputenc}
\usepackage{null}
\def\addbibtoc{
\addcontentsline{toc}{section}{\numberline{\mbox{}}\relax\bibname}
}%end-preamble
\setcounter{page}{1}
\title{SCO, Skunkware, and the Open Source Movement}
\author{Ronald Joe Record
{\ttfamily \url{(rr@sco.com)}},
{\ttfamily Open Source Program Architect},
{\bfseries SCO}}
\date{February 1, 1999}
\abstract{The UNIX technical community has a longstanding tradition of publishing
the source code to programs in order to share technical accomplishments
and facilitate peer review. Examples of this include sendmail, bind,
the X11 graphical windowing system and dozens of USENET newsgroups devoted
to the exchange of source. The recent rise in popularity of the Apache web
server and the Linux operating system have provided a spotlight for
"Open Source" software. How does SCO fit in this picture ? How can SCO
customers take advantage of this type of software ? How can SCO developers
contribute to this movement and leverage the eyes and minds of thousands of programmers on the Internet ?}


\begin{document}
\maketitle

\section{What is Open Source Software ?}



There are a wide variety of phrases used and misused in describing software
whose source is freely available. In fact, a plethora of licensing schemes
are available for such software. Commonly used terms include Open Source,
Free Software, Freeware, Shareware and Public Domain software. Recently Eric 
Raymond has created \nameurl{http://www.opensource.org/}{} where he provides a 
definition of the Open Source certification mark and links to a variety of 
conforming software licenses at \nameurl{http://www.opensource.org/osd.html}{}.

Richard Stallman and the Free Software Foundation have provided an extensive
review of the many categories of software whose source is available at
\nameurl{http://www.fsf.org/philosophy/categories.html}{}. This article will
not attempt to delve into the vagaries and nuances of the specifics of each
of these licenses and categories. The term Open Source Software will be used
to include all software whose source code is freely available and openly 
accessible. This will include Freeware and Shareware when accompanied by
source (e.g. XV) as well as commercial products whose source is freely 
available (e.g. Mini-SQL).




\section{Open Source Components in SCO Products}

Recent months have seen an explosion in Open Source awareness. However, freely
available source code has been with us for dozens of years. Much of the
infrastructure of the Internet is based on Open Source software. Many of
the core components of a UNIX operating environment are Open Source.

Examples of Open Source components in standard UNIX environments include
the mail transport agent {\bfseries sendmail}, the Berkeley Internet Name Domain
{\bfseries BIND}, Dynamic Host Configuration Protocol {\bfseries DHCP},
the InterNetNews server {\bfseries INN} and the X11 graphical windowing system
{\bfseries X}.

Many Open Source components derived from research work done at Universities.
Partly in support of this research the UNIX operating system source has
traditionally been offered to Universities at a minimal licensing fee. When
SCO acquired the rights to early UNIX source (the Mini UNIX operating system; 
the UNIX V6 operating system; the PWB UNIX operating system; and the UNIX V7 
operating system, which also covers Editions 1-5, and the 32V), source licenses
were made available at cost.




\section{What is SCO Skunkware ?}

SCO Skunkware is the generic name for a free collection of software 
prebuilt and prepackaged for installation on SCO systems. The SCO OpenServer
5.0.5 and UnixWare 7 media kits contain an SCO Skunkware CD engineered 
specifically for that operating system.

Distributions are released on CD periodically and a repository of this 
and previous distributions as well as updates and corrections can 
always be found at \nameurl{http://www.sco.com/skunkware}{}. The SCO Skunkware
CD can also be ordered online via \nameurl{http://www.sco.com/offers/}{}.

SCO Skunkware contains a wide variety of software ranging from educational
and experimental research tools to commercial grade software suitable for
use on a production server. 

It is provided for free and is not formally supported by SCO. However, 
Skunkware is undergoing something of a repositioning as previously unsupported
components move into the standard supported product. What has, in the past,
been known as Skunkware will likely continue to exist as a component of a
more traditionally supported Open Source supplement.

The software on the Skunkware CD-ROM is licensed under a variety
of terms. Much of it is licensed under the terms of the 
\nameurl{http://www.sco.com/skunkware/info/gpl.html}{GNU General Public License}.
Some is licensed under the
\nameurl{http://www.sco.com/skunkware/info/lgpl.html}{GNU Library General Public License}.
Other components are licensed under the
\nameurl{http://www.sco.com/skunkware/info/Artistic}{Artistic License}. Many of the components are
"Freeware" with no restrictions on their redistribution while a few components
are "Shareware" meaning the author would like you to try the software and,
if you wish to use it, send her some money. A few components are commercial
products which can be used freely for non-commercial purposes.
Some components simply restrict their use to non-commercial purposes.

To determine the licensing conditions for a particular component, see the
corresponding source in the \nameurl{http://www.sco.com/skunkware/src}{source directory}. With the infrequent exception of SCO proprietary 
code, all Skunkware components are accompanied by the source used to build them.

The Santa Cruz Operation, Inc. and SCO Skunkware are not
related to, affiliated with or licensed by the famous Lockheed
Martin Skunk Works (R), the creator of the F-117 Stealth Fighter,
SR-71, U-2, Venturestar(tm), Darkstar(tm), and other pioneering
air and spacecraft.




\section{Open Source Components in SCO Skunkware}

SCO utilizes Skunkware as a delivery mechanism for Open Source components
which can provide customers with integrated solutions in a wide variety of
emergant enabling technologies and productivity tools. Among these are:
\begin{itemize}
\item {\bfseries The GNU C Compilation system} - perhaps the most widely used 
cross platform C/C++/Objective C and Fortran development environment
\item {\bfseries Mtools} - utilities to access DOS disks and manipulate DOS files 
and directories
\item {\bfseries Industry standard scripting languages} - Perl, Tcl, TclX, Tk, 
Python, BLT, Itcl and Expect
\item {\bfseries Internet/Networking servers and tools} - the latest releases of 
Apache, the world's most widely used commercial grade web server; Squid, a high
performance cacheing proxy server; INN, a complete USENET news server; Enhydra,
a Java application server; IRC, internet relay chat clients and server
\item {\bfseries Editors and text processing tools} - groff, SGML-Tools, TeX,
xcoral, xemacs, ghostscript, vim, xhtml
\item {\bfseries Java applications, servlets, classes and development kits} - 
the Java Servlet Development Kit, the Java Foundation Classes (Swing), the 
Apache JServ server-side Java module, Apache JMeter URL performance meter, 
the Acme labs Java classes and applications, VRwave VRML browser, Jikes Java 
bytecode compiler, Java bytecode editors/debuggers/obfuscators/disassemblers
\item {\bfseries Multi-media content creation and viewing tools} - the GNU Image 
Manipulation Program (Photoshop-like facility), ImageMagick image processing
suite, Xanim animation viewer, MPEG 3 audio encoder and player, MIDI player,
audio editing and conversion tools, graphic file conversion and manipulation
libraries and tools
\item {\bfseries System administration and security tools} - Sentry and Strobe 
port scan detector and optimized TCP port surveyor, Cgiwrap for secure user 
access to CGI, Procdump and Top for information about live processes or core 
image, RPM the Redhat Package Manager
\item {\bfseries Database servers and clients} - MySQL, a threaded light-weight 
powerful SQL relational database management system; Addressbook, an on-line 
rolodex; Mini-SQL, another SQL relational database management system
\item {\bfseries Alternate shells and window managers} - Bash/Zsh/Tcsh, 
WindowMaker, KDE
\end{itemize}


Skunkware also contains a gaggle of games, graphics, eye candy and amusements
including:
\begin{itemize}
\item {\bfseries X11 based adventure and video games} - Xdoom, Xgalaga, Xboing, 
Xpool, Xmame
\item {\bfseries Mathematical recreations and research tools} for exploring chaotic
dynamical systems, iterated function systems, Lyapunov and Mandelbrot sets, ...
\item {\bfseries Miscellaneous fun and interesting stuff} - create graphical 
astrology charts, simulate a fish tank, display your Scottish tartan
\end{itemize}


A full list of hundreds of currently shipping Skunkware components can be 
found at \nameurl{http://www.sco.com/skunkware/components.html}{}




\section{The Open Source Support Model}

Open Source Software and SCO Skunkware have an alternative support model.
Since the source is available, it is often possible for the end-user to
self-support or to easily consult with local experts. Further, due to the
communal and open environment in which the software is developed (see below),
there are often quite active and technically adept online discussion groups,
mailing lists, web sites and ftp download areas for patches and updates.

One drawback to this model is that it is difficult for a vendor to provide
monolithic support for a rapidly diverging Open Source product as the user
base modifies their source code and rebuilds the system. Note that a recent
survey of vendors of an Open Source operating system (Linux) revealed that
there are over 40 commercial variants of Linux. This can pose severe
compatibility, interoperability and support problems.




\section{The Open Source Development Model}

Open Source development teams are rapidly emerging as the dominant force
behind the continuing evolution of computing. Eric Raymond's paper,
The Cathedral and The Bazaar 
(\nameurl{http://www.tuxedo.org/~esr/writings/cathedral-bazaar/}{})
details the philosophy, scope and social organization of this model.
"Release early and often, delegate everything you can, be
open to the point of promiscuity" is a development philosophy sharply
in contradistinction with the conservatively centralized approach
of the traditional development model in which large software projects
were "built like cathedrals, carefully crafted by individual 
wizards or small bands of mages working in splendid isolation, with no 
beta to be released before its time".

SCO is attempting to synthesize these models - carefully protecting
those customers heavily invested in the stability, interoperability
and compatibility offered by the traditional approach while rapidly
deploying emerging technologies and methodologies offered by the bazaar.




\section{Linux Emulation}

SCO recently engaged in an Open Source project which oversees the development
and distribution of lxrun, a Linux emulation system. This open source project
is being incorporated into UnixWare 7 as a supported feature of the operating
system. Additional details on lxrun are available at
\nameurl{http://www.sco.com/skunkware/lxrun/}{}.

The lxrun project is an example of how rapidly an open source project can
evolve. It's also an example of one of the many Skunkware components that
are being absorbed into the standard supported product. 




\section{The Skunkware Submission Process}

If you would like to contribute to the ongoing effort to provide quality
Open Source products to SCO customers:
\begin{itemize}
\item Read the Skunkware FAQ at 
\nameurl{http://www.sco.com/skunkware/faq.html}{}
\item Read the Skunkware submission guidelines at 
\nameurl{http://www.sco.com/skunkware/submission.html}{}
\item Join the polecats mailing list by sending an e-mail message 
to listproc@listproc.sco.com with any subject line and a single line 
in the body of the message:
\begin{tscreen}
\begin{verbatim}
      subscribe polecats-l
\end{verbatim}
\end{tscreen}
\end{itemize}





\section{Author and Contributors}

\nameurl{http://www.ocston.org/~rr}{Ronald Joe Record} has worked for 
\nameurl{http://www.sco.com}{The Santa Cruz Operation} 
for over 15 years.  

Record holds a 
\nameurl{http://www.ocston.org/~rr/PhD/intro.html}{Ph.D.} 
in Mathematics from the 
\nameurl{http://www.ucsc.edu}{University of California}.

David Eyes 
{\ttfamily \url{(davidey@sco.com)}}
contributed to this document in design, review and editorial matters.




\section{About This Document}

This document was created using SGML-Tools 1.0.6 in conjunction with TeX, 
Version 3.14159 (Web2C 7.2) running on an SCO UnixWare 7 platform. 

The source to this document is maintained at
\nameurl{http://www.sco.com/skunkware/sgmldocs/scoworld.sgml}{}.
A Makefile and formatted varieties of this document are also available at
\nameurl{http://www.sco.com/skunkware/sgmldocs/}{}. For instance,
you will find a postscript version at
\nameurl{http://www.sco.com/skunkware/sgmldocs/ps/scoworld.ps}{}.



\end{document}
